\section{Đặt vấn đề}
Ngày nay, xử lý ảnh đang là một lĩnh vực mà rất nhiều người quan tâm và nghiên cứu. Nhờ vào sự phát triển mạnh mẽ của Machine Learning - một lĩnh vực nhỏ của Khoa Học Máy Tính, nó có khả năng tự học hỏi dựa trên dữ liệu đưa vào mà không cần phải được lập trình cụ thể, xử lý ảnh đã và đang được ứng dụng vào nhiều lĩnh vực trong cuộc sống: y tế (X Ray Imaging, PET scan,...), thị giác máy tính (giúp máy tính có thể hiểu, nhận biết đồ vật như con người), các công nghệ nhận dạng (vân tay, khuôn mặt,…)\\

Hiện nay, do sự phát triển của các trang thương mại điện tử - nơi diễn ra sự trao đổi hàng hóa (ví dụ như Shoppe, Lazada,...) ngày càng lớn mạnh nên việc đăng tải các hình ảnh, thông tin của các sản phẩm lên các trang thương mại điện tử đó ngày một phổ biến làm cho lượng dữ liệu trên đó càng trở nên khổng lồ. Chính vì vậy, việc nhận biết và phân loại các sản phẩm đó giúp chúng ta khi muốn tìm kiếm một sản phẩm nào đó dễ dàng hơn.\\     

Các sản phẩm về thời trang như quần áo, giày dép, phụ kiện,... là các sản phẩm hết sức thông dụng trong đời sống con người nên lượng thông tin về các sản phẩm này cũng hết sưc phong phú.Vì thế, việc nhận biết và phân loại các sản phẩm về mặt thời trang cũng có thể coi là một việc cần thiết.\\

\begin{block}{Bài toán}
\begin{itemize}
    \item Input: 
    \begin{itemize}
        \item Hình ảnh của các sản phẩm về thời trang như áo thun, giày, ví, ...
        \item Thông tin của các sản phẩm trên được lưu trong 1 file csv
    \end{itemize}

    \item Bài toán gồm có 3 class: Apparel, Accessories, Footwear

    \item Output: Hệ thống nhận diện và phân loại hình ảnh thuộc lĩnh vực thời trang có khả năng đưa ra được kết quả dự đoán xem hình ảnh đó thuộc class nào.
\end{itemize}
\end{block}